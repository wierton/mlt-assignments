\documentclass[a4paper,UTF8]{article}
\usepackage{ctex}
\usepackage[margin=1.25in]{geometry}
\usepackage{color}
\usepackage{graphicx}
\usepackage{amssymb}
\usepackage{amsmath}
\usepackage{amsthm}
\usepackage{enumerate}
\usepackage{bm}
\usepackage{tcolorbox}
\usepackage{hyperref}
\usepackage{pgfplots}
\usepackage{epsfig}
\usepackage{color}
\usepackage{mdframed}
\usepackage{lipsum}
\newmdtheoremenv{thm-box}{定理}
\newmdtheoremenv{prop-box}{Proposition}
\newmdtheoremenv{def-box}{定义}

\setlength{\evensidemargin}{.25in}
\setlength{\textwidth}{6in}
\setlength{\topmargin}{-0.5in}
\setlength{\topmargin}{-0.5in}
% \setlength{\textheight}{9.5in}
%%%%%%%%%%%%%%%%%%此处用于设置页眉页脚%%%%%%%%%%%%%%%%%%
\usepackage{fancyhdr}                                
\usepackage{lastpage}                                     \usepackage{cite}
\usepackage{layout}                                             
\footskip = 10pt 
\pagestyle{fancy}                    % 设置页眉                 
\lhead{2019年春季}                    
\chead{机器学习理论导引}                                                
% \rhead{第\thepage/\pageref{LastPage}页} 
\rhead{作业三}                                                                                               
\cfoot{\thepage}                                                
\renewcommand{\headrulewidth}{1pt}  			%页眉线宽,设为0可以去页眉线
\setlength{\skip\footins}{0.5cm}    			%脚注与正文的距离           
\renewcommand{\footrulewidth}{0pt}  			%页脚线宽,设为0可以去页脚线

\makeatletter 									%设置双线页眉                                        
\def\headrule{{\if@fancyplain\let\headrulewidth\plainheadrulewidth\fi%
		\hrule\@height 1.0pt \@width\headwidth\vskip1pt	%上面线为1pt粗  
		\hrule\@height 0.5pt\@width\headwidth  			%下面0.5pt粗            
		\vskip-2\headrulewidth\vskip-1pt}      			%两条线的距离1pt        
	\vspace{6mm}}     								%双线与下面正文之间的垂直间距              
\makeatother  

%%%%%%%%%%%%%%%%%%%%%%%%%%%%%%%%%%%%%%%%%%%%%%
\numberwithin{equation}{section}
%\usepackage[thmmarks, amsmath, thref]{ntheorem}
\newtheorem{theorem}{Theorem}
\newtheorem*{myDef}{Definition}
\newtheorem*{mySol}{Solution}
\newtheorem*{myProof}{Proof}
\newtheorem*{myRemark}{备注}

\newcommand{\indep}{\rotatebox[origin=c]{90}{$\models$}}
\newcommand*\diff{\mathop{}\!\mathrm{d}}

\usepackage{multirow}

%--

%--
\begin{document}
\allowdisplaybreaks
\renewcommand{\tilde}{\widetilde}
\renewcommand{\hat}{\widehat}
	\title{机器学习理论导引\\
		作业三}
	\author{欧先飞,DZ1833019,ouxianfei@smail.nju.edu.cn}
	\maketitle

\section{[20pts] Concentration Inequalities}
	\begin{enumerate}[(1)]
		\item \textbf{[5pts]}  请利用高斯函数的性质推导$\Pr[\bar{X}-\mu\geq\epsilon]$上界.
		
		\item \textbf{[5pts]}  请给出Bernstein不等式以$1-\delta$概率表达形式.
		
		\item \textbf{[10pts]} 如果随机变量$X\in[\alpha,\beta]$, 其中$\beta > \alpha$是两个常数。 试证明,随机变量$X$是一个亚高斯随机变量(sub-Gaussian Random Variable),且满足其亚高斯系数$b=(\beta-\alpha)^2/4$, 即对于任意$\lambda > 0$,有下式成立
		\begin{equation}
			\label{eq:def-subgaussian}
			\ln \mathbb{E}[\exp\left\{\lambda (X-\mu) \right\}] \leq \lambda^2(\beta-\alpha)^2/8,
		\end{equation}
		其中$\mu = \mathbb{E}[X]$.		
	\end{enumerate}

\begin{myProof}
~\\
$(1)$ $X \sim N(\mu, \sigma^2) \Longrightarrow \sum_{i=1}^m X_i \sim N(m\mu, m\sigma^2) \Longrightarrow \bar{X} = \frac{1}{m} \sum_{i=1}^m X_i \sim N(\mu, \frac{\sigma^2}{m})$,所以$\Pr(\bar{X} - \mu \ge \epsilon) = \Pr (\bar{X} \ge \mu + \epsilon) = \int_{\mu + \epsilon}^{+\infty} \frac{1}{\sqrt{2\pi \frac{\sigma^2}{m}}} \exp(- \frac{(x-\mu)^2}{2\times\frac{\sigma^2}{m}}) dx$,令$\sigma_0^2 = \frac{\sigma^2}{m}$,于是有:
\begin{eqnarray*}
\Pr (\bar{X} \ge \mu + \epsilon) &=& \int_{\mu + \epsilon}^{+\infty} \frac{1}{\sqrt{2\pi \sigma_0^2}} \exp(- \frac{(x-\mu)^2}{2\sigma_0^2}) dx \\
&=& \int_{\epsilon}^{+\infty} \frac{1}{\sqrt{2\pi \sigma_0^2}} \exp(- \frac{x^2}{2\sigma_0^2}) dx \\
&=& \int_{0}^{+\infty} \frac{1}{\sqrt{2\pi \sigma_0^2}} \exp(- \frac{(x + \epsilon)^2}{2\sigma_0^2}) dx \\
&=& \int_{0}^{+\infty} \frac{1}{\sqrt{2\pi \sigma_0^2}} \exp(- \frac{x^2}{2\sigma_0^2}) \exp(- \frac{\epsilon^2}{2\sigma_0^2}) \exp(- \frac{2x\epsilon}{2\sigma_0^2}) dx \\
&=& \exp(- \frac{\epsilon^2}{2\sigma_0^2}) \int_{0}^{+\infty} \frac{1}{\sqrt{2\pi \sigma_0^2}} \exp(- \frac{x^2}{2\sigma_0^2})  \exp(- \frac{2x\epsilon}{2\sigma_0^2}) dx \\
&\le& \exp(- \frac{\epsilon^2}{2\sigma_0^2}) \int_{0}^{+\infty} \frac{1}{\sqrt{2\pi \sigma_0^2}} \exp(- \frac{x^2}{2\sigma_0^2}) dx \\
&=& \exp(- \frac{\epsilon^2}{2\sigma_0^2}) = \exp(- \frac{m\epsilon^2}{2\sigma^2}) \\
\end{eqnarray*}
~\\
$(2)$ $\Pr(\bar{X_n} \ge \mu + \epsilon) \le \exp(- \frac{n \epsilon^2}{2V + 2b\epsilon}) = \delta  \Longrightarrow \epsilon = \frac{1}{n}(\sqrt{b^2\ln^2\delta - 2nV\ln\delta} - b \ln\delta)$,所以$\Pr(\bar{X_n} \ge \mu + \frac{\sqrt{b^2\ln^2\delta - 2nV\ln\delta} - b \ln \delta}{n}) \le \delta$。

~\\
$(3)$ 令$Y = X - \mu \in [\alpha - \mu, \beta - \mu]$,其中$\mu = E(X)$,记$a = \alpha - \mu$, $b = \beta - \mu$,则$Y \in [a, b]$。令$Z = \frac{Y-a}{b-a}$,则$Y = (b-a)Z + a = Zb + (1 - Z)a$。因为$Z = \frac{Y-a}{b-a} \in [0, 1]$,由Jensen不等式有$e^{\lambda Y} = e^{Z \lambda b + (1-Z) \lambda a} \le Z e^{\lambda b} + (1-Z)e^{\lambda a}$。对不等式两边求取一下期望:
\begin{eqnarray*}
\mathbb{E}(e^{\lambda Y}) &\le& e^{\lambda b} \mathbb{E}(Z) + e^{\lambda a} (1 - \mathbb{E}(Z)) \\
&=& e^{\lambda b} \frac{-a}{b - a} + e ^ {\lambda a}\frac{b}{b - a} \\
&=& e ^ {\lambda a} \left(\frac{b}{b - a} - \frac{a}{b - a}e^{\lambda (b - a)} \right) \\
&=& e ^ {\frac{a}{b - a} \lambda (b - a)} \left(1 + \frac{a}{b - a} - \frac{a}{b - a}e^{\lambda (b - a)} \right) \\
\end{eqnarray*}

对两边求一下对数,则有:
\begin{eqnarray*}
	\ln \mathbb{E}(e^{\lambda Y}) &\le& \frac{a}{b - a} \lambda (b - a) + \ln \left(1 + \frac{a}{b - a} - \frac{a}{b - a}e^{\lambda (b - a)} \right) \\
\end{eqnarray*}

令$x = \lambda (b - a), t = \frac{a}{b - a}$,则$\frac{a}{b - a} \lambda (b - a) + \ln \left(1 + \frac{a}{b - a} - \frac{a}{b - a}e^{\lambda (b - a)} \right) = tx + \ln ( 1 + t - t e^x )$,令$h(x) = tx + \ln ( 1 + t - t e^x )$,则有$h'(x) = t + \frac{-te^x}{1 + t - te^x}$,进一步有$h''(x) = \frac{-te^x(1 + t - te^x) - (-te^x)^2}{(1 + t - te^x)^2} = -\left(\frac{te^x}{1 + t - te^x} \right)-\left(\frac{te^x}{1 + t - te^x} \right)^2 = -\left(\frac{te^x}{1 + t - te^x} + \frac{1}{2} \right)^2 + \frac{1}{4} \le \frac{1}{4}$,由泰勒展开定理知存在$\xi \in [0, x]$使得:
\begin{eqnarray*}
h(x) &=& h(0) + xh'(0) + \frac{x^2}{2} h''(\xi) \\
&=& 0 + x \cdot 0 + \frac{x^2}{2} h''(\xi) \\
&\le&  \frac{x^2}{2} \cdot \frac{1}{4} = \frac{\lambda^2(b-a)^2}{8}
\end{eqnarray*}

代回则有:
\begin{eqnarray*}
	\ln \mathbb{E}(e^{\lambda Y}) = \ln \mathbb{E}(e^{\lambda(X - \mu)}) &\le& h(x) \le \frac{\lambda^2(b-a)^2}{8} = \frac{\lambda^2(\beta-\alpha)^2}{8} \\
\end{eqnarray*}


	~\\
	~\\
	~\\
	~\\
	\qed
\end{myProof}
\newpage

\section{[30pts] McDiarmid Inequality}
\noindent	请利用Chernoff引理证明如下的McDiarmid不等式:
	\begin{thm-box}[McDiarmid Inequality]
	$X_1,X_2,\ldots,X_m$为定义在空间$\mathcal{X}$上的$m$个独立的随机变量. 假设$f\colon \mathcal{X}^m\mapsto \mathbb{R}$是一个关于$X_1,\ldots,X_m$的实值函数, 并满足对任意的$x_1,x_2,\ldots,x_m,x'_i\in\mathcal{X}$, 都有
	\[
	|f(x_1,\ldots,x_i,\ldots,x_m)-f(x_1,\ldots,x'_i,\ldots,x_m)|\leq c_i
	\]
	成立,其中$c_i >0$. 那么对任意$\epsilon>0$, 有
	\[
	\Pr[f(X_1,\ldots,X_m)-\mathbb{E}[f(X_1,\ldots,X_m)]\geq\epsilon]\leq e^{-2\epsilon^2/\sum_{i=1}^mc_i^2}.
	\]
\end{thm-box}
	请写出详细的证明过程。
	\begin{myProof}slides中的证明跳了一些步骤我没能补齐,这里的证明方法参考自一篇stack exchange上的回答\cite{21513}。~\\

记$f(X_{1:m}) = f(X_1, \ldots, X_m)$。令$V_i = \mathbb{E}[f(X_{1:m}) | X_1, \ldots, X_i] - \mathbb{E}[f(X_{1:m}) | X_1, \ldots, X_{i-1}],\,1 < i \le m$,对于边界情况有$V_1 = \mathbb{E}[f(X_{1:m}) | X_1] - \mathbb{E}[f(X_{1:m})]$,易见$f(X_{1:m}) - \mathbb{E}[f(X_{1:m})] = \sum_{i=1}^m V_i$。
由马尔科夫不等式有:
\begin{eqnarray*}
\Pr[f(X_{1:m})-\mathbb{E}[f(X_{1:m})] \geq \epsilon] &=& \Pr[\lambda\left(f(X_{1:m})-\mathbb{E}[f(X_{1:m})]\right) \geq \lambda\epsilon] \\
&=& \Pr[\exp\left\{\lambda\left(f(X_{1:m})-\mathbb{E}[f(X_{1:m})]\right)\right\} \geq \exp(\lambda\epsilon)] \\
&=& \Pr\left[\exp\left\{\lambda\sum_{i=1}^m V_i\right\} \geq \exp(\lambda\epsilon)\right] \\
&\le& e^{-\lambda \epsilon} \mathbb{E}\left[ \exp\left\{\lambda\sum_{i=1}^m V_i\right\} \right]
\end{eqnarray*}

先给$\mathbb{E}\left[ e^{\lambda V_i} \right]$一个上界,令

\begin{eqnarray*}
H_i &= \sup V_i &= \sup_u ( \mathbb{E}[f(X_{1:m}|X_1, \ldots, X_{i-1}, u) ] - \mathbb{E}[f(X_{1:m}|X_1, \ldots, X_{i-1}) ] ) \\
L_i &= \inf V_i &= \inf_v ( \mathbb{E}[f(X_{1:m}|X_1, \ldots, X_{i-1}, v) ] - \mathbb{E}[f(X_{1:m}|X_1, \ldots, X_{i-1}) ] ) \\
\end{eqnarray*}

对两者逐差有:
\begin{eqnarray*}
H_i - L_i &=& \sup_u\mathbb{E}[f(X_{1:m}|X_1, \ldots, X_{i-1}, u) ] - \inf_v\mathbb{E}[f(X_{1:m}|X_1, \ldots, X_{i-1}, v) ]) \\
&=& \int_{-\infty}^{+\infty}\ldots\int_{-\infty}^{+\infty} \sup_u f(x_1, \ldots, x_{i-1}, u, x_{i+1}, \ldots, x_m) f_{X_{i+1}}(x_{i+1}) \ldots f_{X_{m}}(x_m) d x_{i+1} \ldots d x_m \\
&& - \int_{-\infty}^{+\infty}\ldots\int_{-\infty}^{+\infty} \inf_v f(x_1, \ldots, x_{i-1}, v, x_{i+1}, \ldots, x_m) f_{X_{i+1}}(x_{i+1}) \ldots f_{X_{m}}(x_m) d x_{i+1} \ldots d x_m \\
&=& \int_{-\infty}^{+\infty}\ldots\int_{-\infty}^{+\infty} \sup_{u,v} (f(x_1, \ldots, x_{i-1}, u, x_{i+1}, \ldots, x_m) \\
&& - f(x_1, \ldots, x_{i-1}, v, x_{i+1}, \ldots, x_m) ) f_{X_{i+1}}(x_{i+1}) \ldots f_{X_{m}}(x_m) d x_{i+1} \ldots d x_m \\
&\le& \int_{-\infty}^{+\infty}\ldots\int_{-\infty}^{+\infty} c_i f_{X_{i+1}}(x_{i+1}) \ldots f_{X_{m}}(x_m) d x_{i+1} \ldots d x_m \\
&=& c_i
\end{eqnarray*}

又由chernoff引理,有$\ln \mathbb{E}[e^{\lambda V_i}] \le \frac{\lambda^2(H_i - L_i)^2}{8} \le \frac{\lambda^2c_i^2}{8}$,于是:
\begin{eqnarray*}
\mathbb{E}\left[ \exp\left\{\lambda\sum_{i=1}^m V_i\right\} \right] &=& \mathbb{E} \left[ \mathbb{E}\left[ \exp\left\{\lambda\sum_{i=1}^m V_i\right\} | X_1, \ldots, X_{m-1} \right] \right] \\
&=& \mathbb{E} \left[ \exp\left\{\lambda\sum_{i=1}^{m-1} V_i\right\} \mathbb{E}\left[ e^{\lambda V_m} | X_1, \ldots, X_{m-1} \right] \right] \\
&\le& \mathbb{E} \left[ \exp\left\{\lambda\sum_{i=1}^{m-1} V_i\right\} \exp [ \frac{\lambda^2c_m^2}{8} ] \right] \\
&\le& \cdots \le \exp ( \frac{\lambda^2}{8} \sum_{i=1}^m c_i^2 )
\end{eqnarray*}

令$\lambda = \frac{4\epsilon}{\sum_{i=1}^{m} c_i^2}$,则有:
\begin{eqnarray*}
\Pr[f(X_{1:m})-\mathbb{E}[f(X_{1:m})] \geq \epsilon] &\le& e^{-\lambda \epsilon} \mathbb{E}\left[ \exp\left\{\lambda\sum_{i=1}^m V_i\right\} \right] \\
&\le& e^{-\lambda \epsilon} \exp ( \frac{\lambda^2}{8} \sum_{i=1}^m c_i^2 ) \\
&=& e^{-2\epsilon^2/\sum_{i=1}^mc_i^2}
\end{eqnarray*}

	~\\
	~\\
	~\\
	~\\
	\qed
	\end{myProof}

\newpage
\section{[20pts] Basic Inequalities}
	\noindent
	假设$\xi_1,\xi_2,\ldots,\xi_n$是$n$个伯努利随机变量, 且满足$\xi_i\sim B(1/i)$, 设$T_n=\sum_{i=1}^n \xi_i$, 试证明
	\begin{align*}
		\mathbb{E}[T_n] =& \sum_{i=1}^n\frac{1}{i}\geq \ln n. \\
		\mathbb{V}[T_n] =& \sum_{i=2}^n \frac{1}{i}\left(1-\frac{1}{i}\right)\leq \ln n+1.
	\end{align*}
	
	
\begin{myProof}
~\\
$(1)$ $\mathbb{E}(T_n) = \mathbb{E} (\sum_{i=1}{n} \xi_i) = \sum_{i=2}{n} \mathbb{E} (\xi_i) = \sum_{i=1}{n} \frac{1}{i}$,而$\ln n = \int_1^n \frac{1}{x} dx \le \sum_{i=1}^n \frac{1}{i} $,所以有$ \mathbb{E}(T_n) = \sum_{i=1}{n} \frac{1}{i} \ge \ln n$。

~\\
$(2)$令$\mu = \mathbb{E}(T_n)$,
\begin{eqnarray*}
\mathbb{V}(T_n) &=& \mathbb{E}[(T_n - \mathbb{E}(T_n))^2] \\
&=& \mathbb{E}[ T_n^2 - 2\mu T_n + \mu^2 ] \\
&=& \mathbb{E}(T_n^2) - 2\mu\mathbb{E}(T_n) + \mathbb{E}(\mu^2) \\
&=& \mathbb{E}[ \sum_{i=1}^{n} \sum_{j=1}^{n} \xi_i \xi_j ] - \mu^2 \\
&=& \sum_{i=1}^{n} \sum_{j=1}^{n} [ \mathbb{I}_{i=j} \frac{1}{i} + \mathbb{I}_{i\ne j} \frac{1}{ij} ] - \mu^2 \\
&=& \sum_{i=1}^{n} \sum_{j=1}^{n} [ \frac{1}{ij} ] + \sum_{i=1}^{n} [ \frac{1}{i} - \frac{1}{i^2} ] - \mu^2 \\
&=& \sum_{i=1}^{n} \frac{1}{i} \sum_{j=1}^{n} [ \frac{1}{j} ] + \sum_{i=1}^{n} [ \frac{1}{i} - \frac{1}{i^2} ] - \mu^2 \\
&=& \sum_{i=1}^{n} \frac{1}{i}\mu + \sum_{i=1}^{n} [ \frac{1}{i} - \frac{1}{i^2} ] - \mu^2 \\
&=& \mu^2 + \sum_{i=1}^{n} [ \frac{1}{i} - \frac{1}{i^2} ] - \mu^2 \\
&=& \sum_{i=1}^{n} [ \frac{1}{i} - \frac{1}{i^2} ] = \sum_{i=1}^{n} \frac{1}{i} [ 1 - \frac{1}{i} ] \\
&\le& \sum_{i=2}^n \frac{1}{i} < \int_1^n \frac{1}{x} dx = \ln n < \ln n + 1
\end{eqnarray*}
	
	~\\
	~\\
	~\\
	~\\
	\qed
\end{myProof}

\newpage
\section{[30pts] Rademacher Complexity based Margin Bounds}

\noindent 记$S=\{(x_1,y_1),(x_2,y_2),\ldots,(x_m,y_m)\}$是一个大小为$m$的训练集,其中每个样本$(x_i,y_i)$是独立同分布地从分布$\mathcal{D}$中采样得到. 试证明,对任意$\delta > 0$和$\theta > 0$, 任意分类器$H \in \mathcal{C}(\mathcal{H})$至少以$1-\delta$的概率满足:
\begin{equation}
	\label{eq:margin-bound}
	\Pr\nolimits_{(x,y)\sim \mathcal{D}}[yH(x)<0]\leq \sum_{i=1}^{m}\mathbb{I}[y_i H(x_i)\leq\theta]+\frac{2}{\theta}\hat{\mathfrak{R}}_S(\mathcal{H})+3\sqrt{\frac{\ln(1/\delta)}{2m}}.
\end{equation}
其中,函数$\mathbb{I}[\ \cdot \ ]$,仅当$\ \cdot \ $为真时取值为1,否则为0; $\mathcal{C}(\mathcal{H})$为函数空间$\mathcal{H}$的凸包,具体形式为
\[
	\mathcal{C}(\mathcal{H}) = \left\{ H = \sum_{i} \alpha_i h_i: \ h_i \in \mathcal{H}, \alpha_i \geq 0 \mbox{ and } \sum_{i}\alpha_i = 1\right\}.
\]
	
\begin{myProof}
~\\
令$\mathcal{H}_\mathcal{C} = \mathcal{C}(\mathcal{H})$,$\tilde{\mathcal{H}}_\mathcal{C} = \{ (x, y) \mapsto yh(x), h \in \mathcal{H}_\mathcal{C} \}$,记$\Phi_\rho(x) = \mathbb{I}_{x \le 0} + (1 - \frac{x}{\rho})\mathbb{I}_{0 < x \le \rho}$,同时记$\tilde{\mathcal{H}}_{\mathcal{C},\rho} = \Phi_\rho \circ \tilde{\mathcal{H}}_\mathcal{C}$,易见$\tilde{\mathcal{H}}_{\mathcal{C},\rho}$中函数均是$\frac{1}{\rho}-lipshitz$的,且$\Phi_\rho(x) \in [0, 1]$,由slides泛化性一章结论可知:
\begin{equation*}
\mathbb{E}[\Phi_\rho(yh(x))] \le \frac{1}{m}\sum_{i=1}^m \Phi_\rho(yh(x)) + 2 \hat{\mathfrak{R}}(\tilde{\mathcal{H}}_{\mathcal{C},\rho}) + 3 \sqrt{\frac{\ln 1 / \delta}{2m}}
\end{equation*} ~\\
对于要证的不等式左侧,$\Pr_{(x,y)\sim \mathcal{D}}[yH(x)<0] = \mathbb{E}[\mathbb{I}_{yH(x) < 0}] \le \mathbb{E}[ \Phi_\rho(yH(x)) ]$。~\\
对于不等式右侧的第一项,$\frac{1}{m}\sum_{i=1}^m \Phi_\rho(yh(x)) \le \frac{1}{m}\sum_{i=1}^m \mathbb{I}_{yh(x) \le \rho} \le \sum_{i=1}^m \mathbb{I}_{yh(x) \le \rho}$~\\
对于不等式右侧的第二项,
\begin{eqnarray*}
\hat{\mathfrak{R}}(\tilde{\mathcal{H}}_{\mathcal{C},\rho}) &=& \hat{\mathfrak{R}}(\Phi_\rho \circ \tilde{\mathcal{H}}_\mathcal{C}) \le \frac{1}{\rho} \hat{\mathfrak{R}}(\tilde{\mathcal{H}}_\mathcal{C}) \\
&=& \frac{1}{\rho} \mathbb{E}_\sigma [ \sup_{h \in \tilde{\mathcal{H}}_\mathcal{C}} \frac{1}{m} \sum_{i=1}^m \sigma_i h(x_i)y_i ] \\
&=& \frac{1}{\rho} \mathbb{E}_\sigma [ \sup_{h \in \mathcal{H}_\mathcal{C}} \frac{1}{m} \sum_{i=1}^m \sigma_i h(x_i) ] \\
&=& \frac{1}{\rho} \mathbb{E}_\sigma [ \sup_{h_1 \in \mathcal{H}, \ldots, h_d \in \mathcal{H}} \frac{1}{m} \sum_{i=1}^m \sigma_i \sum_{j=1}^d \alpha_j h_j(x_i) ] \\
&=& \frac{1}{\rho} \mathbb{E}_\sigma [ \sum_{j=1}^d \alpha_j \sup_{h_j \in \mathcal{H}} \frac{1}{m} \sum_{i=1}^m \sigma_i h_j(x_i) ] \\
&=& \frac{1}{\rho} \sum_{j=1}^d \alpha_j  \mathbb{E}_\sigma [ \sup_{h_j \in \mathcal{H}} \frac{1}{m} \sum_{i=1}^m \sigma_i h_j(x_i) ] \\
&=& \frac{1}{\rho} \sum_{j=1}^d \alpha_j  \hat{\mathfrak{R}}(\mathcal{H}) \\
&=& \frac{1}{\rho} \hat{\mathfrak{R}}(\mathcal{H}) \\
\end{eqnarray*} ~\\
将上述三个结合,即有
\begin{eqnarray*}
\Pr_{(x,y)\sim \mathcal{D}}[yH(x)<0] &\le& \mathbb{E}[ \Phi_\rho(yH(x)) ] \\
&\le& \frac{1}{m}\sum_{i=1}^m \Phi_\rho(yh(x)) + 2 \hat{\mathfrak{R}}(\tilde{\mathcal{H}}_{\mathcal{C},\rho}) + 3 \sqrt{\frac{\ln 1 / \delta}{2m}} \\
&\le& \sum_{i=1}^m \mathbb{I}_{yh(x) \le \rho} + 2 \frac{1}{\rho} \hat{\mathfrak{R}}(\mathcal{H}) + 3 \sqrt{\frac{\ln 1 / \delta}{2m}} \\
\end{eqnarray*}

	~\\
	~\\
	~\\
	~\\
	\qed
\end{myProof}
\newpage
\bibliographystyle{plain}
\bibliography{refer}
\end{document}