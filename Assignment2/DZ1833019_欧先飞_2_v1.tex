\documentclass[a4paper,UTF8]{article}
\usepackage{ctex}
\usepackage[margin=1.25in]{geometry}
\usepackage{color}
\usepackage{graphicx}
\usepackage{amssymb}
\usepackage{amsmath}
\usepackage{amsthm}
\usepackage{enumerate}
\usepackage{bm}
\usepackage{tcolorbox}
\usepackage{hyperref}
\usepackage{pgfplots}
\usepackage{epsfig}
\usepackage{color}
\usepackage{mdframed}
\usepackage{lipsum}
\newmdtheoremenv{thm-box}{myThm}
\newmdtheoremenv{prop-box}{Proposition}
\newmdtheoremenv{def-box}{定义}

\setlength{\evensidemargin}{.25in}
\setlength{\textwidth}{6in}
\setlength{\topmargin}{-0.5in}
\setlength{\topmargin}{-0.5in}
% \setlength{\textheight}{9.5in}
%%%%%%%%%%%%%%%%%%此处用于设置页眉页脚%%%%%%%%%%%%%%%%%%
\usepackage{fancyhdr}                                
\usepackage{lastpage}                                           
\usepackage{layout}                                             
\footskip = 10pt 
\pagestyle{fancy}                    % 设置页眉                 
\lhead{2019年春季}                    
\chead{机器学习理论导引}                                                
% \rhead{第\thepage/\pageref{LastPage}页} 
\rhead{作业二}                                                                                               
\cfoot{\thepage}                                                
\renewcommand{\headrulewidth}{1pt}  			%页眉线宽,设为0可以去页眉线
\setlength{\skip\footins}{0.5cm}    			%脚注与正文的距离           
\renewcommand{\footrulewidth}{0pt}  			%页脚线宽,设为0可以去页脚线

\makeatletter 									%设置双线页眉                                        
\def\headrule{{\if@fancyplain\let\headrulewidth\plainheadrulewidth\fi%
		\hrule\@height 1.0pt \@width\headwidth\vskip1pt	%上面线为1pt粗  
		\hrule\@height 0.5pt\@width\headwidth  			%下面0.5pt粗            
		\vskip-2\headrulewidth\vskip-1pt}      			%两条线的距离1pt        
	\vspace{6mm}}     								%双线与下面正文之间的垂直间距              
\makeatother  

%%%%%%%%%%%%%%%%%%%%%%%%%%%%%%%%%%%%%%%%%%%%%%
\numberwithin{equation}{section}
%\usepackage[thmmarks, amsmath, thref]{ntheorem}
\newtheorem{theorem}{Theorem}
\newtheorem*{myDef}{Definition}
\newtheorem*{mySol}{Solution}
\newtheorem*{myProof}{Proof}
\newtheorem*{myRemark}{备注}
\newcommand{\indep}{\rotatebox[origin=c]{90}{$\models$}}
\newcommand*\diff{\mathop{}\!\mathrm{d}}

\usepackage{multirow}

%--

%--
\begin{document}
\renewcommand{\tilde}{\widetilde}
\renewcommand{\hat}{\widehat}

	\title{机器学习理论导引\\
		作业二}
	\author{DZ1833019,欧先飞,ouxianfei@smail.nju.edu.cn}
	\maketitle
	
	
	\section{[30pts] Generalization}
	
	机器学习中,我们总会通过先验知识对假设空间进行限制. 例如SVM中使用的典型超平面族$\mathcal{H} = \left\lbrace \boldsymbol{x} \mapsto \boldsymbol{w}^{\top}\boldsymbol{x} : \left\| \boldsymbol{w} \right\| \leq \Lambda  \right\rbrace $(见讲义的定理5.10). 因为虽然大的假设空间更可能包含目标概念,但对应的学习难度即样本复杂度也随之增大,从而导致泛化性变差.
	
	
	\begin{enumerate}[(1)]
		\item \textbf{[10pts]}  试通过VC维的泛化误差界来解释对假设空间进行限制的合理性。
		
		\item \textbf{[10pts]}  试通过Rademacher的泛化误差界来解释对假设空间进行限制的合理性。
		
		\item \textbf{[10pts]}	二者的泛化误差界哪个更紧?为什么?
	\end{enumerate}
	
	
	\begin{myProof}~\\
$(1)$ 因为$\Pr[E(h) \le \hat{E}(h) + \left( \frac{em}{d}\right)^d  + \sqrt{\frac{\ln 1 / \delta}{2m}}] \ge 1 - \delta$,而当对假设空间进行限制时,相应的该假设空间的VC维也倾向于降低$($因为假设空间更小了,VC维无论如何不会变大$)$,从该不等式获得的泛化误差的上界也会相应的变小,从而学习算法的泛化性能可以更好。
~\\
$(2)$ 因为$\Pr[E(h) \le \hat{E}(h) + \mathfrak{R}_m(\mathcal{H})  + \sqrt{\frac{\ln 1 / \delta}{2m}}] \ge 1 - \delta$,当假设空间受限之后,$\mathfrak{R}_m(\mathcal{H}) = E_{D^m, \sigma}[\sup_{h \in \mathcal{H}}\frac{1}{m} \sum_{i=1}^m \sigma_i h(x_i) ]$式中$\sup_{h \in \mathcal{H}}\frac{1}{m} \sum_{i=1}^m \sigma_i h(x_i)$也将倾向于更小$($无论如何不会更大$)$,从而由该不等式获得的泛化误差的上界也会更小。
~\\
$(3)$ 基于Rademacher的更紧,因为$\mathfrak{R}_m(\mathcal{H}) \le \left( \frac{em}{d}\right)^d$。
		
		~\\
		~\\
		~\\
		~\\
		\qed
	\end{myProof}
	\newpage
	\section{[20pts] Stability}
	\noindent
	
	\begin{enumerate}[ {(}1{)}]
		\item \textbf{[10pts]} 为了应对未知的测试情况,实际机器学习算法在选择超参数取值时,通常通过交叉验证的方式来估计泛化能力。请讨论留一法交叉验证估计学习算法泛化能力的合理性(从稳定性的角度进行分析;留一法交叉验证参考周志华《机器学习》26页)。
		\item \textbf{[10pts]} 假设讲义中定理6.1所需的条件均满足,如果算法非常稳定, 即$\beta\rightarrow 0$, 是否可以通过同样的分析得到优于$\mathcal{O}(1/\sqrt{m})$的泛化界?
	\end{enumerate}
	
	
	\begin{myProof}~\\
$(1)$ 从稳定性角度来看,$\Pr[ \ell(\mathfrak{L}, D) \le \ell_{loo} ( \mathfrak{L}, D) + \beta + (4m\beta + M) \sqrt{\frac{\ln 1 / \delta}{2m}} ] \ge 1 - \delta$,当使用留一法计算所得的损失较小时,整体的泛化损失在同等概率下也会更小,所以留一法用于评估模型的泛化能力是比较合理的。
~\\
$(2)$ slides中推导所得的泛化损失的界$\Pr[ \ell(\mathfrak{L}, D) \le \hat{\ell} ( \mathfrak{L}, D) + \gamma + (2m\gamma + M) \sqrt{\frac{\ln 1 / \delta}{2m}} ] \ge 1 - \delta$,容易发现该上界由三个渐进项构成$\mathcal{O}(\gamma)$、$\mathcal{O}(\gamma \sqrt{m})$和$\mathcal{O}({\frac{1}{\sqrt{m}}})$,所以无论$\gamma$取什么样的渐进函数,或者直接取0,该不等式的泛化界都不会优于$\mathcal{O}({\frac{1}{\sqrt{m}}})$。

		
		~\\
		~\\
		~\\
		~\\
		\qed
	\end{myProof}
	
	\newpage
	\section{[20pts] Optimality of Bayes Classifier}
	对任意定义在$\mathcal{X}\times\{0,1\}$上的概率分布$\mathcal{D}$,考虑所有分类器$g:\mathcal X \mapsto \{0,1\}$,特定的,记$f_{\mathcal{D}}$为Bayes分类器,其定义如下:
	\[
	f_{\mathcal{D}} = 
	\begin{cases}
	1,&\mbox{if }\Pr[y=1|x]\geq 1/2\\
	0,&\mbox{otherwise}
	\end{cases}
	\]
	试证明,Bayes分类器$f_{\mathcal D}$是最优的,即对任何分类器$g$,有$R(f_{\mathcal D}) \leq R(g)$,其中$R(g)$是分类器$g$在未知数据分布$\mathcal{D}$的泛化误差,$R(g) = \Pr_{(x,y) \sim \mathcal{D}}[g(x) \neq y]$。
	
	\begin{myProof}
首先证明$\Pr[h(x)=k, y=k | x = x_0] = \Pr[h(x) = k | x = x_0] \Pr[y=k | x = x_0], k \in {0, 1} $,由于当$x$给定时,$h(x)$只可能是确定的0或者1,所以直接对$h(x_0)$的取值进行讨论。
~\\
假设$k=0$,如果$h(x_0)=0$,那么$\Pr[h(x)=k, y=k | x = x_0] = \Pr[h(x) = k | x = x_0] \Pr[y=k | x = x_0] = \Pr[y=0 | x = x_0]$,如果$h(x_0 = 1)$,那么$\Pr[h(x)=k, y=k | x = x_0] = \Pr[h(x) = k | x = x_0] \Pr[y=k | x = x_0] = 0$,所以当$k=0$时,$\Pr[h(x)=k, y=k | x = x_0] = \Pr[h(x) = k | x = x_0] \Pr[y=k | x = x_0]$,同理当$k=1$时等式也成立。继而有:
\begin{eqnarray*}
\Pr[h(x) = y | x = x_0] &=& \Sigma _ {k \in \mathcal{Y}} \Pr[h(x) = k |x = x_0] \Pr [y = k | x = x_0] \\
&=& \Sigma _ {k \in \mathcal{Y}} \mathbb{I}[h(x_0) = k] \Pr [y = k | x = x_0] \\
&=& \mathbb{I}[h(x_0) = 0] \Pr [y = 0 | x = x_0] + \mathbb{I}[h(x_0) = 1] \Pr [y = 1 | x = x_0] 
\end{eqnarray*}
~\\
然后对$f_D$与$g$进行逐差,令$\Delta = \Pr[f_D(x) = y | x = x_0] - \Pr[g(x) = y | x = x_0]$,则有:
\begin{eqnarray*}
	\Delta &=& \Pr [y = 0 | x = x_0] (\mathbb{I}[f_D(x_0) = 0] - \mathbb{I}[g(x_0) = 0]) \\
	&& + \Pr [y = 1 | x = x_0] (\mathbb{I}[f_D(x_0) = 1] - \mathbb{I}[g(x_0) = 1]) \\
&=& (1 - \Pr[y=1 | x = x_0])(\mathbb{I}[g(x_0) = 1] - \mathbb{I}[f_D(x_0) = 1]) + \\
&& \Pr [y = 1 | x = x_0] (\mathbb{I}[f_D(x_0) = 1] - \mathbb{I}[g(x_0) = 1]) \\
&=& (2\Pr [y = 1 | x = x_0] -1) (\mathbb{I}[f_D(x_0) = 1] - \mathbb{I}[g(x_0) = 1]) 
\end{eqnarray*}
当$\Pr [y = 1 | x = x_0] \ge \frac{1}{2}$时,由$f_D$的定义有$\mathbb{I}[f_D(x_0)=1] = 1$,又因为$\mathbb{I}[g(x_0) = 1] \le 1$,所以$\Delta \ge 0$。
当$\Pr [y = 1 | x = x_0] < \frac{1}{2}$时,由$f_D$的定义有$\mathbb{I}[f_D(x_0)=1] = 0$,又因为$\mathbb{I}[g(x_0) = 1] \ge 0$,所以$\Delta \ge 0$。
综上,对于任意的$g$,始终有$\Delta = \Pr[f_D(x) = y | x = x_0] \ge \Pr[g(x) = y | x = x_0]$,其等价于$\Delta = \Pr[f_D(x) \ne y | x = x_0] \le \Pr[g(x) \ne y | x = x_0]$,也就是$R(f_D) \le R(g)$,所以Bayes分类器$f_D$是最优的。
		
		~\\
		~\\
		~\\
		~\\	
		\qed
	\end{myProof}
	\newpage
	
	\section{[30pts] SVM with Squared Hinge Loss Function}
	在支持向量机(Support Vector Machine,SVM)的实际使用中,人们经常采用平方hinge损失函数(squared hinge loss function)。记损失函数为$\ell: \mathcal{Y}'\times \mathcal{Y} \mapsto \mathbb{R}_+$, 其中$\mathcal{Y}' \subset \mathbb{R}$且$\mathcal{Y} = \{-1,+1\}$,平方hinge损失函数的定义可写为
	\begin{equation}
	\label{eq:squared-hinge-loss}
	\ell(y',y) = ([1-yy']_+)^2,
	\end{equation}
	其中符号$[x]_+$表示取$x$的非负部分,即$[x]_+ = x$如果$x\geq 0$;否则$[x]_+ = 0$. 本题目中,我们采用第六讲中所讲授的\emph{稳定性}工具对平方hinge损失SVM的泛化性进行分析。
	
	\begin{enumerate}[ {(}1{)}]
		\item \textbf{[10pts]} 假设对于任意的分类器$h\in \mathcal{H}$及样本$x\in \mathcal{X}$,均有$\lvert h(x) \rvert \leq M$,试证明平方hinge损失函数是有界的,并给出上界。
		\item \textbf{[20pts]} 试利用\emph{稳定性}分析工具推导基于平方hinge损失SVM的泛化界。请给出严格的结论表述和具体的推导过程。
	\end{enumerate}
	
	\begin{myProof}~\\
$(1)$首先容易证明$[a+b]_+ \le |a| + |b|$,所以有$\ell(h(x), y) = ([1 - h(x)y]_+)^2 \le (1 + |h(x)|)^2 \le (1 + M)^2$。
~\\
$(2)$由于$|\ell(h(a), y) - \ell(h(b), y)| = | ([1 - h(a)y]_+)^2 - ([1 - h(b)y]_+)^2 | \le | (1 + |h(a)y|)^2 - (1 + |h(b)y|)^2 | = | (1 + |h(a)|)^2 - (1 + |h(b)|)^2 | = | (h(a) +  h(b) + 2) | |h(a) - h(b)| \le (2M + 2) |h(a) - h(b)|$,所以损失函数$\ell$对于假设空间$\mathcal{H}$是$\sigma$-可容许的,其中$\sigma=2M+2$。
然后由slides上的命题6.1知SVMs具有$\gamma$-稳定性,$\gamma < \frac{4(M+1)^2r^2}{m\lambda}$,所以$\ell(\mathcal{L, D}) \le \hat{\ell}(\mathcal{L}, D) + \gamma + (2m\gamma + M) \sqrt{\frac{\ln 1/\delta}{2m}} < \hat{\ell}(\mathcal{L}, D) + \frac{4(M+1)^2r^2}{m\lambda} + (\frac{8(M+1)^2r^2}{\lambda} + M) \sqrt{\ln 1/\delta}{2m}$,以至少$1 - \delta$的概率成立。
		

	~\\
	~\\
	~\\
	~\\
	\qed
	\end{myProof}
	
\end{document}