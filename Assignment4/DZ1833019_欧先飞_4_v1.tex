\documentclass[a4paper,UTF8]{article}
\usepackage{ctex}
\usepackage[margin=1.25in]{geometry}
\usepackage{color}
\usepackage{graphicx}
\usepackage{amssymb}
\usepackage{amsmath}
\usepackage{amsthm}
\usepackage{enumerate}
\usepackage{bm}
\usepackage{tcolorbox}
\usepackage{hyperref}
\usepackage{pgfplots}
\usepackage{epsfig}
\usepackage{color}
\usepackage{mdframed}
\usepackage{lipsum}
% add the new command "\norm"
\usepackage{mathtools}
\let\oldnorm\norm   % <-- Store original \norm as \oldnorm
\let\norm\undefined % <-- "Undefine" \norm
\DeclarePairedDelimiter\norm{\lVert}{\rVert}
\DeclarePairedDelimiter\abs{\lvert}{\rvert}
\newmdtheoremenv{thm-box}{myThm}
\newmdtheoremenv{prop-box}{Proposition}
\newmdtheoremenv{def-box}{定义}

\setlength{\evensidemargin}{.25in}
\setlength{\textwidth}{6in}
\setlength{\topmargin}{-0.5in}
\setlength{\topmargin}{-0.5in}
% \setlength{\textheight}{9.5in}
%%%%%%%%%%%%%%%%%%此处用于设置页眉页脚%%%%%%%%%%%%%%%%%%
\usepackage{fancyhdr}                                
\usepackage{lastpage}                                           
\usepackage{layout}                                             
\footskip = 10pt 
\pagestyle{fancy}                    % 设置页眉                 
\lhead{2019年春季}                    
\chead{机器学习理论导引}                                                
% \rhead{第\thepage/\pageref{LastPage}页} 
\rhead{作业四}                                                                                               
\cfoot{\thepage}                                                
\renewcommand{\headrulewidth}{1pt}  			%页眉线宽,设为0可以去页眉线
\setlength{\skip\footins}{0.5cm}    			%脚注与正文的距离           
\renewcommand{\footrulewidth}{0pt}  			%页脚线宽,设为0可以去页脚线

\makeatletter 									%设置双线页眉                                        
\def\headrule{{\if@fancyplain\let\headrulewidth\plainheadrulewidth\fi%
		\hrule\@height 1.0pt \@width\headwidth\vskip1pt	%上面线为1pt粗  
		\hrule\@height 0.5pt\@width\headwidth  			%下面0.5pt粗            
		\vskip-2\headrulewidth\vskip-1pt}      			%两条线的距离1pt        
	\vspace{6mm}}     								%双线与下面正文之间的垂直间距              
\makeatother  

%%%%%%%%%%%%%%%%%%%%%%%%%%%%%%%%%%%%%%%%%%%%%%
\numberwithin{equation}{section}
%\usepackage[thmmarks, amsmath, thref]{ntheorem}
\newtheorem{theorem}{Theorem}
\newtheorem*{myDef}{Definition}
\newtheorem*{mySol}{Solution}
\newtheorem*{myAnalysis}{Analysis}
\newtheorem*{myProof}{Proof}
\newtheorem*{myRemark}{备注}
\renewcommand{\tilde}{\mathbf{w}idetilde}
\renewcommand{\hat}{\mathbf{w}idehat}
\newcommand{\indep}{\rotatebox[origin=c]{90}{$\models$}}
\newcommand*\diff{\mathop{}\!\mathrm{d}}

\usepackage{multirow}

%--

%--
\begin{document}
	\title{机器学习理论导引\\
		作业四}
	\author{姓名,邮箱}
	\maketitle
	
	% \section*{作业提交注意事项}
	% \begin{tcolorbox}
	% 	\begin{enumerate}
	% 		\item[(1)] 请严格参照课程网站作业提交方法一节提交作业,未按照要求提交作业,或提交作业格式不正确,将会被扣除部分作业分数;
	% 		\item[(2)] 截止时间后不接收作业,本次作业记零分;
	% 		\item[(3)] 本课程要求大家遵守学术诚信,请关注课程主页的详细说明。
	% 	\end{enumerate}
	% \end{tcolorbox}
	
	% \newpage

\section{[15pts] Conjugate Functions}
\noindent 推导下面函数的共轭函数
\[
	f(x)=\log(1+\exp(-x)).
\]
	
\begin{myProof}~\\
	~\\
	~\\
	~\\
	~\\	
	\qed
\end{myProof}

\newpage
\section{[15pts] Projection}
\noindent 对于凸集$\mathcal{W}$,试证明投影操作$\Pi_{\mathcal{W}}(\cdot)$是不扩展的,即
\[
\norm{\Pi_{\mathcal{W}}(\mathbf{x})-\Pi_{\mathcal{W}}(\mathbf{y})} \leq \norm{\mathbf{x} - \mathbf{y}}, \ \forall \mathbf{x}, \mathbf{y}.
\]

\begin{myProof}~\\
	~\\
	~\\
	~\\
	~\\	
	\qed
\end{myProof}

\newpage
\section{[20pts] Gradient Descent with Decaying Step Size}
\noindent 分析采用衰减步长时梯度下降(GD)的收敛速率。具体而言,考虑
\[
\eta_t=O\left(\frac{1}{\sqrt{t}}\right)
\]

\begin{myAnalysis}~\\
	~\\
	~\\
	~\\
	~\\	
	~\\
\end{myAnalysis}


\newpage
\section{[50pts] Stochastic Optimization}
\noindent 考虑随机优化问题
\[
\min_{\mathbf{w} \in \mathcal{W}} \ F(\mathbf{w})= \mathbb{E}_{\xi} \left[ f(\mathbf{w}, \xi) \right]
\]
其中目标函数是$\lambda$强凸的,也就是
\begin{equation}\label{eqn:strong}
F(\mathbf{w}) + \langle \nabla F(\mathbf{w}), \mathbf{w}'-\mathbf{w}  \rangle + \frac{\lambda}{2} \|\mathbf{w}'-\mathbf{w}\|_2^2 \leq F(\mathbf{w}'), \ \forall \mathbf{w}, \mathbf{w}' \in \mathcal{W}.
\end{equation}
试分析采用$\eta_t=O(1/[\lambda t])$的随机梯度下降算法的额外风险。
\begin{enumerate}
  \item[(1)] \textbf{[25pts]} 证明期望意义上的额外风险为$O(\log T/T)$。\\
  提示:该问题非常简单,可以将步长设置为$\eta_t=1/[\lambda t]$。然后,参考[1]中定理1的证明,得到
\[
 \sum_{t=1}^T F(\mathbf{w}_{t}) - T F(\mathbf{w}) \leq \frac{G^2}{2 \lambda} (\log T+1) + \sum_{t=1}^T \langle \nabla F(\mathbf{w}_t)-\mathbf{g}_t, \mathbf{w}_t - \mathbf{w} \rangle.
\]
接下来只需要求期望,化简即可。
  \item[(2)] \textbf{[25pts]} 证明$O(\log T/T)$的收敛速率同样以大概率成立。\\
第一种途径:可以将步长设置为$\eta_t=2/[\lambda t])$。然后,参考[1]中定义1的证明,得到
\[
 \sum_{t=1}^T F(\mathbf{w}_{t}) - T F(\mathbf{w}) \leq    \frac{G^2}{\lambda}(\log T+1) +\sum_{t=1}^T \langle \nabla F(\mathbf{w}_t)-\mathbf{g}_t, \mathbf{w}_t - \mathbf{w} \rangle- \frac{\lambda}{4} \sum_{t=1}^T \|\mathbf{w}_t - \mathbf{w}\|^2.
\]
参考讲义公式(21)的推导过程,我们知道以至少$1-\delta$的概率
 \[
 \sum_{t=1}^T \langle \nabla F(\mathbf{w}_t)-\mathbf{g}_t, \mathbf{w}_t - \mathbf{w} \rangle \leq  2\sqrt{ 4 G^2  \log \frac{m }{\delta} \left( \sum_{t=1}^T \|\mathbf{w}_t - \mathbf{w}\|^2\right)} +  \frac{8G^2}{3\lambda}   \log \frac{m }{\delta} + \frac{4G^2}{\lambda} .
\]
最后对结果化简即可。\\
第二种途径:参考论文[2]。
\end{enumerate}
\begin{myProof}~\\
	~\\
	~\\
	~\\
	~\\	
	\qed
\end{myProof}
\newpage
\section*{Reference}
\begin{enumerate}[ {[}1{]}]
\item E. Hazan, A. Agarwal, and S. Kale. Logarithmic regret algorithms for online convex optimization. \textit{Machine Learning}, 69(2-3):169-192, 2007.
\item S. M. Kakade and A. Tewari. On the generalization ability of online strongly convex programming algorithms. In \textit{Advances in Neural Information Processing Systems 21}, pages 801-808, 2009.
\end{enumerate}
\end{document}