\documentclass[a4paper,UTF8]{article}
\usepackage{ctex}
\usepackage[margin=1.25in]{geometry}
\usepackage{color}
\usepackage{graphicx}
\usepackage{amssymb}
\usepackage{amsmath}
\usepackage{amsthm}
\usepackage{enumerate}
\usepackage{bm}
\usepackage{tcolorbox}
\usepackage{hyperref}
\usepackage{algorithm}  
\usepackage{algorithmicx}  
\usepackage{algpseudocode}
\usepackage{pgfplots}
\usepackage{epsfig}
\usepackage{color}
\usepackage{mdframed}
\usepackage{lipsum}
% add the new command "\norm"
\usepackage{mathtools}
\let\oldnorm\norm   % <-- Store original \norm as \oldnorm
\let\norm\undefined % <-- "Undefine" \norm
\DeclarePairedDelimiter\norm{\lVert}{\rVert}
\DeclarePairedDelimiter\abs{\lvert}{\rvert}
\newmdtheoremenv{thm-box}{myThm}
\newmdtheoremenv{prop-box}{Proposition}
\newmdtheoremenv{def-box}{定义}

\setlength{\evensidemargin}{.25in}
\setlength{\textwidth}{6in}
\setlength{\topmargin}{-0.5in}
\setlength{\topmargin}{-0.5in}
% \setlength{\textheight}{9.5in}
%%%%%%%%%%%%%%%%%%此处用于设置页眉页脚%%%%%%%%%%%%%%%%%%
\usepackage{fancyhdr}                                
\usepackage{lastpage}                                           
\usepackage{layout}                                             
\footskip = 10pt 
\pagestyle{fancy}                    % 设置页眉                 
\lhead{2019年春季}                    
\chead{机器学习理论导引}                                                
% \rhead{第\thepage/\pageref{LastPage}页} 
\rhead{作业五}                                                                                               
\cfoot{\thepage}                                                
\renewcommand{\headrulewidth}{1pt}  			%页眉线宽,设为0可以去页眉线
\setlength{\skip\footins}{0.5cm}    			%脚注与正文的距离           
\renewcommand{\footrulewidth}{0pt}  			%页脚线宽,设为0可以去页脚线

\makeatletter 									%设置双线页眉                                        
\def\headrule{{\if@fancyplain\let\headrulewidth\plainheadrulewidth\fi%
		\hrule\@height 1.0pt \@width\headwidth\vskip1pt	%上面线为1pt粗  
		\hrule\@height 0.5pt\@width\headwidth  			%下面0.5pt粗            
		\vskip-2\headrulewidth\vskip-1pt}      			%两条线的距离1pt        
	\vspace{6mm}}     								%双线与下面正文之间的垂直间距              
\makeatother  

%%%%%%%%%%%%%%%%%%%%%%%%%%%%%%%%%%%%%%%%%%%%%%
\numberwithin{equation}{section}
%\usepackage[thmmarks, amsmath, thref]{ntheorem}
\newtheorem{theorem}{Theorem}
\newtheorem*{myDef}{Definition}
\newtheorem*{mySol}{Solution}
\newtheorem*{myAnalysis}{Analysis}
\newtheorem*{myProof}{Proof}
\newtheorem*{myRemark}{备注}
\newcommand{\indep}{\rotatebox[origin=c]{90}{$\models$}}
\newcommand*\diff{\mathop{}\!\mathrm{d}}

\usepackage{multirow}

%--

%--
\begin{document}
\renewcommand{\tilde}{\mathbf{w}idetilde}
\renewcommand{\hat}{\mathbf{w}idehat}

	\title{机器学习理论导引\\
		作业五}
	\author{DZ1833019,欧先飞,ouxianfei@smail.nju.edu.cn}
	\maketitle
	
	% \section*{作业提交注意事项}
	% \begin{tcolorbox}
	% 	\begin{enumerate}
	% 		\item[(1)] 请严格参照课程网站作业提交方法一节提交作业,未按照要求提交作业,或提交作业格式不正确,将会被扣除部分作业分数;
	% 		\item[(2)] 截止时间后不接收作业,本次作业记零分;
	% 		\item[(3)] 本课程要求大家遵守学术诚信,请关注课程主页的详细说明。
	% 	\end{enumerate}
	% \end{tcolorbox}
	
	% \newpage

\section{[20pts] Online Regression}
\noindent [在线回归]考虑在线岭回归:
\begin{itemize}
  \item 每一时刻$t$,学习器选择分类面$\mathbf{w}_t \in \{\mathbf{w}| \norm{\mathbf{w}} \leq D\} \subseteq \mathbb{R}^d$;
  \item 学习器观测到样本和标记$(\mathbf{x}_t,y_t)$,并遭受损失
  \[
  f_t(\mathbf{w}_t)=(y_t -\mathbf{w}_t^\top \mathbf{x}_t)^2 + \frac{\lambda}{2} \|\mathbf{w}_t\|^2
  \]
其中$\lambda>0$是给定的正则化参数,$\norm{\mathbf{x}_t} \leq R$,$|y_t| \leq DR$.
\end{itemize}
请问学习器应该采用什么算法更新分类面$\mathbf{w}_t$? 采用该算法之后,学习器的遗憾是多少?\\
要求:描述算法时,需要给出梯度的具体计算公式、如何设置步长。

\begin{myAnalysis}~\\
给定$(\mathbf{x}_t, y_t)$,则$\frac{\partial f_t(\mathbf{w}_t)}{\mathbf{w}_t} = \frac{\partial}{\partial \mathbf{w}_t}\left[ (y_t - \mathbf{w}_t^\mathrm{T} \mathbf{x}_t)^2 + \frac{\lambda}{2} \norm{\mathbf{w}_t}_2^2 \right] = 2(\mathbf{w}_t^T \mathbf{x} - y_t) \mathbf{x}_t + \lambda \mathbf{w}_t$,由题意有$\norm{\mathbf{w}_t} \le D$,$\norm{\mathbf{x}_t} \le R$, $|y_t| \le DR$。

所以有$\norm{\frac{\partial f_t(\mathbf{w}_t)}{\mathbf{w}_t}} \le 2|\mathbf{w}_t^\mathrm{T} \mathbf{x}_t - y_t| \norm{\mathbf{x}_t} + \lambda \norm{\mathbf{w}_t} \le 2(DR + DR) R + \lambda D = 4DR^2 + \lambda D$。
参考pdf第32页定理3.3,将学学习率设置为$\eta_t = \frac{2R}{4DR^2 + \lambda D} \sqrt{t}$,则由定理的结论,可以获得算法的遗憾界为:
\[
\sum_{t=1}^T f_t(\mathbf{w}_t) - \min_{\mathbf{w} \in \mathbb{W}} \sum_{t=1}^T f_t(\mathbf{w}) \le \frac{3}{2} 2R [4DR^2 + \lambda D] \sqrt{T} = 3DR[4R^2 + \lambda] \sqrt{T} = O(\sqrt{T})
\]
而相应的算法为:
\begin{algorithm}  
	\caption{Online Regression}  
	\begin{algorithmic}[1] %每行显示行号  
		\State{randomly initialize $\mathbf{w}_1 \in \mathcal{W}$}
		\For{t = 1...T}
		  \State{$\mathbf{w}_{t+1}' = \mathbf{w}_t - \frac{2R}{(4DR^2 + \lambda D) \sqrt{t}} \left[ 2(\mathbf{w}_t^\mathrm{T} \mathbf{x}_t - y_t) \mathbf{x}_t + \lambda \mathbf{w}_t \right] $}
		  \State{ $\mathbf{w}_{t+1} = \frac{D}{\max\{ D, \norm{\mathbf{w}_{t+1}'} \} } \mathbf{w}_{t+1}' $ }
		\EndFor  
	\end{algorithmic} 
\end{algorithm}  
\end{myAnalysis}

\newpage
\section{[25pts] Logistic Regression}
\noindent [在线分类] 考虑有约束的逻辑回归问题:
\begin{itemize}
  \item 每一时刻$t$,学习器选择分类面$\mathbf{w}_t \in \{\mathbf{w}| \|\mathbf{w}\| \leq D\} \subseteq \mathbb{R}^d$;
  \item 学习器观测到样本和标记$(\mathbf{x}_t,y_t)$,并遭受损失
  \[
  f_t(\mathbf{w}_t)=\log\left( 1+ \exp(-y_t \mathbf{w}_t^\top \mathbf{x}_t)\right)
  \]
其中$\|\mathbf{x}_t\| \leq R$,$y_t \in \{+1,-1\}$.
\end{itemize}
请问学习器应该采用什么算法更新分类面$\mathbf{w}_t$?采用该算法之后,学习器的遗憾是多少?\\
要求:描述算法时,需要给出梯度的具体计算公式、如何设置步长。

\begin{myAnalysis}~\\

给定$(\mathbf{x}_t, y_t)$,则有$\frac{\partial f_t(\mathbf{w}_t)}{\partial \mathbf{w}_t} = \frac{\partial}{\partial \mathbf{w}_t} [ \ln (1 + e^{-y_t\mathbf{w}_t^\mathrm{T} \mathbf{x}_t}) ] = \frac{e^{-y_t \mathbf{w}_t^\mathrm{T} \mathbf{x}_t}}{1 + e^{-y_t \mathbf{w}_t^\mathrm{T} \mathbf{x}_t}} (-y_t \mathbf{x}_t) $,进而可以获得其梯度的上界:$\norm{\frac{\partial f_t(\mathbf{w}_t)}{\partial \mathbf{w}_t}} \le \norm{-y_t \mathbf{x}_t} \le R$。参考定理3.3可以知道,将步长设置为$\eta_t = \frac{2R}{R\sqrt{t}} = \frac{2}{\sqrt{t}}$,可以获得目标算法如下:

\begin{algorithm}  
	\caption{Logistic Regression}  
	\begin{algorithmic}[1] %每行显示行号  
		\State{randomly initialize $\mathbf{w}_1 \in \mathcal{W}$}
		\For{t = 1...T}
		\State{$\mathbf{w}_{t+1}' = \mathbf{w}_t - \frac{2}{\sqrt{t}} \frac{e^{-y_t \mathbf{w}_t^\mathrm{T} \mathbf{x}_t}}{1 + e^{-y_t \mathbf{w}_t^\mathrm{T} \mathbf{x}_t}} (-y_t \mathbf{x}_t) $}
		\State{ $\mathbf{w}_{t+1} = \frac{D}{\max\{ D, \norm{\mathbf{w}_{t+1}'} \} } \mathbf{w}_{t+1}' $ }
		\EndFor  
	\end{algorithmic} 
\end{algorithm}

由pdf第32页定理3.3的结论,我们可以获得如下的遗憾界:
\[
\sum_{t=1}^T f_t(\mathbf{w}_t) - \min_{\mathbf{w} \in \mathbb{W}} \sum_{t=1}^T f_t(\mathbf{w}) \le \frac{3}{2} 2R \times R \sqrt{T} = 3R^2 \sqrt{T} = O(\sqrt{T})
\]

由于$\log(1 + \exp(-y_t \mathbf{w}_t^\mathrm{T} \mathbf{x}_t))$是$\alpha$指数凹的,所以采用在线牛顿法可以获得更好的渐进界,参考pdf第35页定理3.5的证明,使用算法如下:
\begin{algorithm}  
	\caption{ONS}  
	\begin{algorithmic}[1] %每行显示行号  
		\State{randomly initialize $\mathbf{w}_1 \in \mathcal{W}$}
		\State{$A_0 = \frac{4}{\beta^2D^2} I$}
		\For{t = 1...T}
		\State{$A_t = A_{t-1} + \nabla f_t(\mathbf{w}_t) \nabla f_t(\mathbf{w}_t) ^ \mathrm{T} $}
		\State{$\mathbf{w}_{t+1}' = \mathbf{w}_t - \frac{1}{\beta} A_t^{-1} \nabla f_t(\mathbf{w}_t) $}
		\State{ $\mathbf{w}_{t+1} = {\arg\min}_{\mathbf{w} \in \mathcal{W}} (\mathbf{w} - \mathbf{w}_{t+1}')^\mathrm{T} A_t (\mathbf{w} - \mathbf{w}_{t+1}') $ }
		\EndFor  
	\end{algorithmic} 
\end{algorithm}

其中$\beta$是输入参数,按照定理3.5的证明,在线牛顿法可以获得$\frac{1}{2\beta}(1 + d \log T) = O(\log T)$的渐进界。

\end{myAnalysis}

\newpage
\section{[20pts] Exponentially Concave}
\noindent 对于二次可微函数$f(\cdot):\mathcal{W} \mapsto \mathbb{R}$,它是$\alpha$指数凹的充要条件是什么?\\
提示:利用函数$f$的梯度和海森矩阵建立充要条件。
\begin{myAnalysis}~\\
	
若函数$f(\mathbf{w})$是$\alpha$指数凹的,则意味着$\exp(-\alpha f(\mathbf{w}))$是凹函数,而这等价于$\exp(-\alpha f(\mathbf{w}))$的海森矩阵半负定。

将$\exp(-\alpha f(\mathbf{w}))$的海森矩阵记为$H$,并记$\mathbf{w} = (w_1, w_2, \ldots, w_d)^\mathrm{T}$,那么有:
\begin{eqnarray*}
H_{ij} &=& \frac{\partial^2 \exp(-\alpha f(\mathbf{w}))} { \partial w_i \partial w_j } \\
&=& \frac{\partial}{\partial w_i} \left[ \frac{\partial \exp(-\alpha f(\mathbf{w}))} { \partial w_j } \right] \\
&=& \frac{\partial}{\partial w_i} \left[ \exp(-\alpha f(\mathbf{w})) (-\alpha \frac{\partial f(\mathbf{w})}{\partial w_j}) \right] \\
&=& \exp(-\alpha f(\mathbf{w})) (-\alpha \frac{\partial f(\mathbf{w})}{\partial w_i}) (-\alpha \frac{\partial f(\mathbf{w})}{\partial w_j}) + \exp(-\alpha f(\mathbf{w})) (-\alpha) \frac{\partial^2f(\mathbf{w})}{\partial w_i w_j} \\
&=& \alpha^2 \exp(-\alpha f(\mathbf{w})) \frac{\partial f(\mathbf{w})}{\partial w_i} \frac{\partial f(\mathbf{w})}{\partial w_j} - \alpha \exp(-\alpha f(\mathbf{w})) \frac{\partial^2 f(\mathbf{w})}{\partial w_i \partial w_j}
\end{eqnarray*}

所以$H = \alpha^2 \exp(-\alpha f(\mathbf{w})) \nabla f(\mathbf{w}) \nabla f(\mathbf{w}) ^ \mathbf{T} - \alpha \exp(-\alpha f(\mathbf{w})) \nabla^2 f(\mathbf{w})$,由定义知:f是$\alpha$指数凹的等价于H半负定。

	~\\	
	~\\
\end{myAnalysis}

\newpage
\section{[25pts] Online-to-Batch Conversation}
\noindent 对于在线凸优化问题,假设$f_1,\ldots,f_T$是从同一分布$\mathcal{P}$独立采样得到。
假设随机函数$f_1,\ldots,f_T$为凸,其定义域$\mathcal{W}$直径小于$D$,梯度的范数小于$G$,根据讲义介绍,在线梯度下降有如下遗憾界:
\begin{equation} \label{eqn:regret}
\sum_{t=1}^T f_t(\mathbf{w}_{t}) -  \sum_{t=1}^T f_t(\mathbf{w}) \leq \frac{3 DG}{2} \sqrt{T}
\end{equation}
令$\bar{\mathbf{w}}=\frac{1}{T} \sum_{t=1}^T \mathbf{w}_t$,$F(\cdot)=\mathbb{E}_{f \sim \mathcal{P}}[f(\cdot)]$。在遗憾界(\ref{eqn:regret})的基础上,证明以大概率
\[
F(\bar{\mathbf{w}}) -  F(\mathbf{w}) = O\left( \frac{1}{\sqrt{T}}\right).
\]
提示:假设随机函数$f$有界,然后利用讲义定理2.2 (针对鞅的Azuma不等式)
\begin{myProof}~\\
	
首先对$F(\bar{\mathbf{w}}) - F(\mathbf{w})$进行变换:
\begin{eqnarray*}
F(\bar{\mathbf{w}}) - F(\mathbf{w}) &=& F(\frac{1}{T} \sum_{t=1}^T \mathbf{w}_t) - F(\mathbf{w}) \\
&\le& \frac{1}{T} \sum_{t=1}^T F(\mathbf{w}_t) - F(\mathbf{w}) \\
&=& \frac{1}{T} \sum_{t=1}^T [F(\mathbf{w}_t) - F(\mathbf{w}) ] \\
&=& \frac{1}{T} \sum_{t=1}^T [ F(\mathbf{w}_t) - f_t(\mathbf{w}_t) + f_t(\mathbf{w}_t) - f_t(\mathbf{w}) + f_t(\mathbf{w}) - F(\mathbf{w}) ] \\
&=& \frac{1}{T} \sum_{t=1}^T [ F(\mathbf{w}_t) - f_t(\mathbf{w}_t)] + \frac{1}{T} \sum_{t=1}^T [ f_t(\mathbf{w}) - F(\mathbf{w}) ] + \frac{1}{T} \sum_{t=1}^T [ f_t(\mathbf{w}_t) - f_t(\mathbf{w}) ]
\end{eqnarray*}

由题目中\ref{eqn:regret}知,上述不等式右边最后一项的上界为$\frac{3DG}{2\sqrt{T}}$,而由于$\mathbb{E} [ f_t(\mathbf{w}_t) ] = F(\mathbf{w}_t)$、$\mathbb{E} [ f_t(\mathbf{w}) ] = F(\mathbf{w})$,所以上述不等式右边前两项,从$t=1$到$T$构成分别构成两个鞅差序列。假设$|f_t(\cdot)| \le c$,则对鞅差序列中每一项有$|F(\mathbf{w}_t) - f_t(\mathbf{w}_t)| \le 2c$,进而由Azuma不等式可以得到:
$\Pr \left[ \sum_{t=1}^T \left( F(\mathbf{w}_t) - f_t(\mathbf{w}_t) \right) \ge \epsilon \right] \le \exp( - \frac{\epsilon^2}{8Tc^2} )$,令$\exp( - \frac{\epsilon^2}{8Tc^2} ) = \delta$可以得到:
\[
\Pr\left[\sum_{t=1}^T \left( F(\mathbf{w}_t) - f_t(\mathbf{w}_t) \right) \ge \sqrt{8Tc^2\ln \frac{1}{\delta}}  \right] \le \delta
\]
对于不等式右边第二项同理可得$\Pr\left[\sum_{t=1}^T \left( f_t(\mathbf{w}_t - F(\mathbf{w}_t)) \right) \ge \sqrt{8Tc^2\ln \frac{1}{\delta}}  \right] \le \delta$。将这两个结论代入上述不等式,则可以得到$\Pr\left[ F(\bar{\mathbf{w}}) - F(\mathbf{w}) \le \frac{2}{T}\sqrt{8Tc^2\ln \frac{1}{\delta}} + \frac{3DG}{2\sqrt{T}}\right] \ge 1 - \delta$,而$\frac{2}{T}\sqrt{8Tc^2\ln \frac{1}{\delta}} + \frac{3DG}{2\sqrt{T}} = 4c\sqrt{\frac{\ln 1 / \delta}{T}} + \frac{3DG}{2\sqrt{T}}  = O(\frac{1}{\sqrt{T}})$。

	~\\
	~\\	
	\qed
\end{myProof}

\newpage
\section{[10pts] Comparison with SGD}
\noindent 在讲义定理2.1中,我们证明了SGD同样可以达到
\[
F(\bar{\mathbf{w}}) -  F(\mathbf{w}) =O\left( \frac{1}{\sqrt{T}}\right).
\]
请问SGD的证明过程和习题4的证明过程有什么异同?
\begin{myAnalysis}~\\
$(1)$两者都利用凸函数的条件对当前解与最优解的上界进行评估,并利用等式$\langle \mathbf{a} - \mathbf{b}, \mathbf{a} - \mathbf{c} \rangle = \frac{1}{2} ( \norm{\mathbf{a}-\mathbf{b}}_2^2 + \norm{\mathbf{a}-\mathbf{c}}_2^2 - \norm{\mathbf{b}-\mathbf{c}}_2^2 )$将上界转化为逐差的形式,然后对T项累加以放缩上界。
~\\
$(2)$不同的地方在于SGD的证明过程直接对函数空间$\mathcal{F}$的期望$F$进行放缩,而习题4中的证明中所利用到的遗憾界是从采样得到的函数$f_t$出发开始放缩。
	~\\
	~\\	
	~\\
\end{myAnalysis}


\end{document}
